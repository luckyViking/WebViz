\documentclass{scrartcl}
\usepackage{hyperref}

\title{Explainable SVM}
\subtitle{Online Visualizations for Teaching}

% FIXME emails look weird?
\author{
  Hendrik Pfaff, Luca Jordan, Lukas Atkinson, Yasemin Er
  \\
  {\normalsize\ttfamily \{hendrik.pfaff,gian.jordan,atkinson,yasemin-\}@stud.fra-uas.de}
}

\date{Summer Term 2020}

\publishers{Frankfurt University of Applied Sciences}

\subject{Project Documentation}

\begin{document}
\maketitle

\section{Introduction}

In order to teach algorithms, interactive visualizations can be a powerful tool.
We therefore developed an interactive, web-based tool
to explain how Support Vector Machines (SVM) work.

\section{Support Vector Machines}

% TODO: provide some mathematical background

\section{Installation}

The visualizations are implemented in client-side JavaScript,
so implementation is simple:
the project files just have to be served via an HTTP server
and opened in any modern browser.
There are no server-side programs that have to be installed.

For local use, Python's built-in web server is a convenient choice:

\begin{enumerate}
\item start the server with \verb|python3 -m http.server -b localhost 4321|
\item open the start page in the browser: \url{http://localhost:4321/views/}
\end{enumerate}

While the user-accessible pages are under \verb|/views/|,
necessary resources are served under
\verb|/data/|, \verb|/css/|, and \verb|/js/|.

A live version of the project can also be accessed at
\url{https://luckyviking.github.io/WebViz/views/}.

\section{Implemented Views}

The primary tool is provided under \verb|/views/plotly.html|.
Other tools were used for debugging or tried out alternative approaches:

\begin{itemize}
\item \verb|/views/scatterplot.html|
\item \verb|/views/different_datasets.html|
\item \verb|/views/iris_shapes.html|
\item \verb|/views/gui.html|
\item \verb|/views/plotly.html|
\end{itemize}

\section{User Manual}

\section{Challenges}

\section{Conclusion}

\end{document}